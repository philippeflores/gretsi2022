\documentclass{gretsi}
%% Selectionnez ensuite les paquets que vous utilisez,
%% par suppression ou adjonction d'un caractere %
%% en debut de ligne (mise en commentaire).
%% --------------------------------------------------------------
%% UTILISATION DE CARACTERES ACCENTUES AU CLAVIER ?
%% (le codage du clavier depend du systeme d'exploitation)
% \usepackage[applemac]{inputenc} % MacOS
% \usepackage[ansinew]{inputenc}  % Windows ANSI
% \usepackage[cp437]{inputenc}    % DOS, page de code 437
% \usepackage[cp850]{inputenc}    % DOS, page de code 850
% \usepackage[cp852]{inputenc}    % DOS, page de code 852
% \usepackage[cp865]{inputenc}    % DOS, page de code 865
% \usepackage[latin1]{inputenc}   % UNIX, codage ISO 8859-1
% \usepackage[decmulti]{inputenc} % VMS
% \usepackage[next]{inputenc}
% \usepackage[latin2]{inputenc}
% \usepackage[latin3]{inputenc}
%% --------------------------------------------------------------
%% REGLES DE TYPOGRAPHIE FRANCAISES ?
% \usepackage[french]{babel}   % "babel.sty" + "french.sty"
\usepackage[english,francais]{babel} % "babel.sty"
% \usepackage{french}                  % "french.sty"
\usepackage{times}			% ajout times le 30 mai 2003
 
%% --------------------------------------------------------------
%% CODAGE DE POLICES ?
%% Si votre moteur Latex est francise, il est conseille
%% d'utiliser le codage de police T1 pour faciliter la c�sure,
%% si vous disposez de ces polices (DC/EC)
\usepackage[T1]{fontenc}
%% ==============================================================

\usepackage{amsmath,epsfig,hyperref,bbm,amsmath,stmaryrd,amssymb,setspace,multirow,scalerel,stackengine,dsfont,ragged2e,chngcntr,graphicx,subcaption,caption}

%%%%%%%%%%%%%%%%%% Notations Mathématiques %%%%%%%%%%%%%%%%%
% TENSORS 

\newcommand{\tens}[1]{\boldsymbol{\mathcal{#1}}} % use calligraphics for tensors
\newcommand{\tenselem}[2]{\mathcal{#1}_{#2}} % entry of a tensor

% MATRICES
\newcommand{\matr}[1]{\boldsymbol{#1}} % use capitals for matrices
\newcommand{\matrelem}[2]{#1_{#2}} % entry of a matrix

% VECTORS
\newcommand{\vect}[1]{\boldsymbol{#1}} % use lowercases for vectors
\newcommand{\vectelem}[2]{#1_#2} % entry of a vector

% SETS, SPACES, VARIETIE
\newcommand{\nset}[1]{\ifmmode\left\llbracket#1\right\rrbracket\else\fi} % ENTIRE SET

%% OPERATORS %%%%%%%%%%%%%%%%%%%%%%%%%
\newcommand{\out}{\mathop{\circ}}              % tensor outer product
\DeclareMathOperator*{\Out}{\scalerel*{\out}{\sum}}  % Big khatri rao product

\newcommand{\khatri}{\odot}                 % Khatri-Rao product
\DeclareMathOperator*{\Khatri}{\scalerel*{\khatri}{\sum}}  % Big khatri rao product

\newcommand{\hadam}{\boldsymbol{\ast}}           % Hadamard product
\DeclareMathOperator*{\Hadam}{\scalerel*{\hadam}{\sum}}   % Big Hadamard Product

\newcommand{\kron}{\mathop{\otimes}}            % Kronecker product
\newcommand{\T}{{\sf T}}                  % transposition

\newcommand{\trace}[1]{\mathop{\operator trace}\{#1\}}
\newcommand{\diag}[1]{\mathop{\operator diag}\{#1\}}  % a vector
\DeclareMathOperator*{\argmin}{argmin} 
\DeclareMathOperator*{\p}{p} 
\DeclareMathOperator*{\Card}{Card} 
\DeclareMathOperator*{\Error}{Err} 
\newcommand{\Err}[1]{\Error\!\!\!\!\!\!\!\quad_{#1}}

\newcommand{\vectorize}{\mathop{\operator vec}}
\newcommand{\unfold}[2]{\matr{#1}_{(#2)}}
\newcommand{\factor}[2]{\matr{#1}\!^{(\!#2\!)}}
%%%%%%%%%%%%%%%%%%%%%%%%%%%%%%%%%%%%%%%%%%%%%%%%%%%%%%%%%%%%%%

%%%%%%%% Chapeaux adaptatif à la taille du mot %%%%%%%%%
\stackMath
\newcommand\reallywidehat[1]{%
\savestack{\tmpbox}{\stretchto{%
 \scaleto{%
  \scalerel*[\widthof{\ensuremath{#1}}]{\kern-.6pt\bigwedge\kern-.6pt}%
  {\rule[-\textheight/2]{1ex}{\textheight}}
 }{\textheight}% 
}{0.5ex}}%
\stackon[1pt]{#1}{\tmpbox}%
}
\parskip 1ex
%%%%%%%%%%%%%%%%%%%%%%%%%%%%%%%%%%%%%%%%%%%%%%%%%%%%%%%%%

%%%%%%%% Dots more compact %%%%%%%%%
%\newcommand\ldots{\ifmmode\ldot\!\ldot\!\ldot\else\makebox[1em][c]{\ldot\hfil\ldot\hfil\ldot}\fi}
%%%%%%%%%%%%%%%%%%%%%%%%%%%%%%%%%%%%

% %%%%%%%% Vertical Triples %%%%%%%%%
% \newcommand{\triples}[3]{$\left\{\!\!\!\!\!\begin{array}{c}
%  #1 \\ #2 \\ #3 \\
% \end{array}\!\!\!\!\!\right\}$}
% %%%%%%%%%%%%%%%%%%%%%%%%%%%%%%%%%%%%

%%%%%%%% Vertical Triples %%%%%%%%%
\newcommand{\triples}[3]{$(#1,#2,#3)$}
%%%%%%%%%%%%%%%%%%%%%%%%%%%%%%%%%%%%


\titre{Factorisation couplée de tenseurs pour l'analyse de données de cytométrie en flux}

\auteur{\coord{Philippe}{Flores}{1},
        \coord{Konstantin}{Usevich}{2},
    \coord{David}{Brie}{1}}

\adresse{\affil{1}{Université de Lorraine, Centre de Recherche en Automatique de Nancy \\
         Campus Sciences BP 70239, 54506 Vandoeuvre-lès-Nancy, France}
         \affil{2}{CNRS, Centre de Recherche en Automatique de Nancy \\
         Campus Sciences BP 70239, 54506 Vandoeuvre-lès-Nancy, France}}

%% Si tous les auteurs ont la m�me adresse %%%%%%%%%%%%%%%%%%%%%%%%%%%%%%%%%%%%
%                                                                             %
%   \auteur{\coord{Michel}{Dupont}{},                                         %
%           \coord{Marcel}{Dupond}{},                                         %
%           \coord{Michelle}{Durand}{},                                       %
%           \coord{Marcelle}{Durand}{}}                                       %
%                                                                             %
%   \adresse{\affil{}{Laboratoire Traitement des Signaux et des Images \\     %
%     1 rue de la Science, BP 00000, 99999 Nouvelleville Cedex 00, France}}   %
%                                                                             %
%%%%%%%%%%%%%%%%%%%%%%%%%%%%%%%%%%%%%%%%%%%%%%%%%%%%%%%%%%%%%%%%%%%%%%%%%%%%%%%

\email{philippe.flores@univ-lorraine.fr,
konstantin.usevich@univ-lorraine.fr\\
david.brie@univ-lorraine.fr\vspace{-0.8cm}}

\resumefrancais{Dans ce papier, nous proposons une nouvelle méthode d'analyse automatique de données de cytométrie en flux. Grâce à une modélisation de la distribution par une combinaison de distributions plus simples, nous reformulons le problème comme une factorisation tensorielle couplée de marginales 3D. Pour réduire les coûts de calcul, nous utilisons des stratégies de couplage partiel. Nous proposons aussi un regroupement des termes de rang 1 ainsi qu'un nouvel outil de visualisation de résultats. Nous montrons l'utilité de ladite méthode avec des données simulées et des données réelles.}

\resumeanglais{In this paper, we propose a new method for automated flow cytometry data analysis. By modeling the distribution as a mixture of simpler distributions, we can reformulate the problem as a coupled tensor approximation of 3D-marginals. In order to reduce the computational load, we use partially coupled strategies. We also propose a grouping of rank-1 components together with a new visualization of the results. We demonstrate the usefulness of the proposed methodology on simulated and real data. \vspace{-0.5cm}}

\begin{document}

\maketitle


\section{Introduction}
\vspace{-0.5cm}
La cytométrie en flux (CMF) est une des techniques d'analyse de cellules biologiques les plus utilisées dans de nombreux domaines comme l'agriculture, la médecine ou la biologie \cite{Saeys2016}. Son application principale reste l'immunologie où elle améliore la connaissance du système immunitaire \cite{Perfetto2004} en permettant aux biologistes de rechercher des populations de cellules rares.

% Dans un cytomètre, un flux de cellules est créé et illuminé par des lasers. La lumière ré-émise est séparée vers des photodétecteurs mesurant des plages spécifiques de longueurs d'ondes, plages étant conçues pour détecter des propriétés biologiques particulières. Avant de passer dans le cytomètre, les cellules sont teintées avec des fluorochromes qui émettent de la lumière dans les longueurs d'ondes associées à un photodétecteur ; la réponse de fluorescence des cellules permet d'identifier et de quantifier la présence d'un marqueur spécifique dans un échantillon de cellules.

% Lors de la préparation d'une expérience de cytométrie, les protéines ou gènes recherchés dans un groupe de cellules sont choisis et sont associés à des fluorochromes permettant d'améliorer la séparation entre les cellules qui présentent ou non ces protéines ou gènes. Les cellules passent alors dans un appareil appelé cytomètre qui crée un flux de cellules individuelles qui passe aux travers de lasers. La lumière ré-émise est séparée vers des photodétecteurs, chaque photodétecteur détectant la lumière associée à un fluorochrome et donc à un marqueur biologique.

D'un point de vue analyse de données, un cytomètre produit un nuage de points dans un espace à $M$ dimensions. Le but est de séparer et identifier les différentes populations de cellules. L'analyse conventionnelle, basée sur une succession d'étapes en 2 dimensions, devient incomplètes, subjectives et coûteuse en temps plus le nombre de paramètres augmente. De ce fait, des méthodes automatiques sont apparues \cite{Qiu2011,Amir2013,VanGassen2015} mais sont coûteuses et peuvent difficilement être appliquées à de grands ensemble de données. De plus, ces méthodes mènent à des résultats insatisfaisants pour des populations de cellules rares et ont des visualisations limitées en grandes dimensions.

Dans ce papier, nous présentons une nouvelle méthode probabiliste appelée CTFlowHD. Pour faire face à la malédiction de la dimension, nous supposons un modèle bayésien naïf pour la densité conjointe multivariée. Ainsi, estimer l'histogramme en $M$ dimensions revient à estimer les facteurs d'un modèle tensoriel CP \cite{Harshman1970} dont la complexité demeure linéaire avec le nombre de dimensions. En suivant \cite{Kargas2018}, l'estimation du modèle CP d'ordre $M$ est formulé comme un problème d'approximation couplée des marginales 3D. Pour diminuer encore les coûts de calcul, nous proposons de considérer seulement une partie des marginales 3D lors du couplage. Enfin, pour améliorer l'interprétation de nos résultats, nous introduisons une étape de clustering supplémentaire qui classe directement les facteurs de rang 1.

\vspace{-0.8cm}
\section{Modèle bayésien naïf pour \\l'estimation de densité de probabilité}
\vspace{-0.2cm}
\subsection{Estimation de densité multivariée} 
\vspace{-0.2cm}
Soit $\vect{x} = \left(X_1,\ldots,X_M\right)$ un vecteur aléatoire prenant des valeurs dans $I_1\times\ldots\times I_M$, où $I_m = \left[x_0^{(m)}, x_K^{(m)}\right]$. En supposant que les lignes de la matrice $\matr{X}$ notées $\vect{x}_{n,:}$ sont des réalisations de $\vect{x}$, notre but est d'estimer la densité de probabilité multivariée (PDF) $\p(X_1,\ldots,X_M)$ du vecteur aléatoire $\vect{x}$ à partir de la matrice d'observation $\matr{X}$. Une approche possible d'estimation de densités est de considérer un histogramme en $M$ dimensions. Dans ce cas, chaque intervalle $I_m$ est séparé en $K$ intervalles égaux allant de $\Delta_1^{(m)} = \left[x_0^{(m)}, x_1^{(m)}\right]$ à $\Delta_K^{(m)} = \left[x_{K-1}^{(m)}, x_K^{(m)}\right]$. Cet histogramme, noté $\tens{H}$, peut être interprété comme une PDF jointe discrétisée.

\begin{align}
    \tens{H} & =
     \Pr(X_1\!\in\!\Delta_{k_1}^{(\!1\!)}, \ldots, X_M\!\in\!\Delta_{k_M}^{(\!M\!)}) \label{eq:linkHistoPDF} \\ 
    & = \int_{X\!_1\in\Delta_{k_1}^{(\!1\!)}}\!\!\!\!\!\!\!\!\!\!\!\!\cdots\hspace{0.2cm}\int_{X\!_M\in\Delta_{k_M}^{(\!M\!)}} \p(X_1,\ldots,X\!_M) dX\!_1\ldots dX\!_M \notag
\end{align}

Pour estimer l'histogramme à partir de $\matr{X}$, les échantillons sont décomptés dans chaque intervalle en $M$ dimensions :

\begin{equation}
    \tens{H} \approx \frac{1}{N}\Card\!\left\{n\!\in\!\nset{1,N} \Big| \vect{x}_n \!\in\! \Delta_{k_1}^{(\!1\!)}\times \ldots\times \Delta_{k_M}^{(\!M\!)} \right\}
\end{equation}

Cependant, cette approche nécessite un nombre d'échantillons qui grandit de manière exponentielle avec le nombre de dimensions. Pour donner un ordre de grandeur avec $M = 8$ et $K = 20$, l'histogramme est décrit par $K^M \approx 10^10$ valeurs et requiert encore plus d'échantillons pour obtenir une estimation précise. Cela est du à la malédiction de la dimension. Pour pallier ce problème, nous suivons l'approche de \cite{Kargas2018} qui utilise un Modèle bayésien Naïf (MBN) dont la complexité demeure linéaire avec le nombre de dimension.

Le Modèle bayésien Naïf (MBN) introduit une variable latente discrète $L$, tel que les éléments de $\vect{x}$ sont conditionnellement indépendant par rapport à $L$ : \begin{equation}
    \label{eq:nbmcont}
    \p(X_1, \ldots, X_M) = \sum\limits_{r=1}^R \Pr(L\!=\!r) \prod\limits_{m=1}^M \p\!\left(X_m | L\!=\!r\right)
\end{equation}
En transposant \eqref{eq:nbmcont} dans \eqref{eq:linkHistoPDF}, nous obtenons que le MBN correspond à une Décomposition Canonyque Polyadique (CPD) d'ordre $M$ \cite{Kolda2009} de $\tens{H}$ \cite{Kargas2018}. \begin{align}
    & \tens{H} \approx \sum\limits_{r=1}^R \Pr(L\!=\! r) \prod\limits_{m=1}^M \Pr\left(X_m \in \Delta_{k_m}^{(\!m\!)} \Big|L\!=\!r\right) \label{eq:nbm} \\
    & \approx \!\nset{\!\vect{\lambda}\!;\! \factor{A}{1}\!,\!\ldots\!,\!\factor{A}{M}} \! = \!\sum\limits_{r=1}^R \vect{\lambda}_r \vect{a}_r^{(\!1\!)} \out \ldots \out \vect{a}_r^{(\!M\!)}. \notag 
\end{align}
Dans ce modèle, $R$ représente le nombre de composante et le rang de la décomposition de $\tens{H}$. De plus, les matrices facteurs $\factor{A}{m} \!=\! \left(\vect{a}_1^{(\!m\!)} \cdots \vect{a}_R^{(\!m\!)}\right) \in \mathbb{R}^{I_m\times R}$ et le vecteur $\vect{\lambda} \in \mathbb{R}^R$ doivent satisfaire les conditions de non-négativité : $\vect{\lambda}>0$, $\factor{A}{m}>0$, et les contraintes de simplexes : $\mathbbm{1}^\T\vect{\lambda}=1$, $\mathbbm{1}^\T\factor{A}{m}=\mathbbm{1}^\T$.

\section{Factorisation tensorielle couplée}

\subsection{Couplage total de tenseurs}

L'idée du couplage de factorisations tensorielles est d'obtenir un MBN en $M$ dimensions grâce à des MBN marginalisés \cite{Kargas2018}. En pratique, les histogrammes 3D sont obtenues facilement tout en gardant les propriétés d'unicités tensorielles que n'ont pas les matrices. Soit $\left( X_i,X_j,X_k\right)$ un triplet de variables aléatoire de $\vect{x}$, le MBN \eqref{eq:nbm} peut être marginaliser pour obtenir un modèle d'ordre 3 qui approche l'histogramme 3D $\tens{H}_{ijk}$.
\begin{equation}
    \label{eq:nbm3D}
    \tens{H}_{ijk} \approx \!\nset{\!\vect{\lambda}\!;\! \factor{A}{i}\!,\!\factor{A}{j}\!,\!\factor{A}{k}\!}
\end{equation} 
Pour estimer les facteurs du MBN, nous considérons l'ensemble de tous les triplets possibles $\mathcal{T}\! = \!\left\{ \!\left(i,\!j,\!k\right)\!\in\! \nset{1,M}^3 \Big| i\!<\!j\!<\!k\! \right\}$ et nous résolvons le problème d'optimisation : \begin{align}
    \footnotesize
    \centering 
    & \reallywidehat{\vect{\lambda}}, \reallywidehat{\factor{A}{1}}, \ldots, \reallywidehat{\factor{A}{M}} = \!\!\!\!\!\!\min_{\vect{\lambda},\factor{A}{1},\ldots, \factor{A}{M}} \!\!\!\!\sum_{(i,j,k)\in \mathcal{T}} \!\!\!\!\!\left|\!\left| \tens{F}_{ijk} \!-\!\! \nset{\!\vect{\lambda}\!;\!\factor{A}{i}\!,\! \factor{A}{j}\!,\! \factor{A}{k}}\!\right|\!\right|_F^2 \notag \\
    & \text{s.t.} \quad \vect{\lambda}\geq0, \factor{A}{m}\geq0, \mathbbm{1}^{\T}\vect{\lambda}=1, \mathbbm{1}^{\T} \factor{A}{m} = \mathbbm{1}^\T, \label{eq:ctfOptim}
\end{align} appelé factorisation tensorielle couplée totalement. \eqref{eq:ctfOptim} est résolu avec une procédure d'AO-ADMM couplée \cite{Kargas2018}.

\subsection{Conditions d'identifiabilité}

Les décompositions tensorielles possèdent des conditions d'unicité (identifiabilité) fortes \cite{Kolda2009}. En particulier, si tous les $\tens{H}_{ijk}$ sont individuellement génériquement identifiables, c'est-à-dire si $R<\frac{3K-2}{2}$, alors le tenseur de probabilité $\tens{H}$ est aussi identifiable. Cependant, comme beaucoup de $\tens{H}_{ijk}$ partagent des facteurs communs, les conditions d'indentifiabilité peuvent être significativement augmentées. En supposant $M\leq K$, $\tens{H}$ est génériquement identifiable si $R \leq K(M-2)$ \cite{Kargas2018}.

Il faut noter que ces résultats d'identifiabilité dérivent de conditions non-bruitée (décomposition exacte) et sont formulées pour des cas de matrices facteurs réelles (possiblement non-negatives). En pratique, comme le nombre d'échantillons est limité, seuls les $\tens{H}_{ijk}$ bruités sont disponibles ce qui mène à un problème d'approximation de tenseur de rang faible. En ajoutant les contraintes de non-négativité sur les facteurs est avantageux vu que cela assure l'existence et l'unicité de l'approximation tensorielle de rang faible, voir \cite{Qi2016}.

Enfin, si on y regarde de plus près, la preuve des résultats d'identifiabilité de \cite{Kargas2018} révèle que seul l'identifiabilité d'une extension du tenseur à une partition spécifique des variables est requise. En d'autres termes, seul un nombre limité de triplets (définis par la partition des $M$ variables) est nécessaire pour assurer l'identifiabilité. Cette idée est développée dans la sous-section suivante pour réduire les coûts de calculs de la factorisation couplée de tenseurs.

\subsection{Couplage partiel de tenseurs}

Dans \eqref{eq:ctfOptim}, tous les triplets possibles sont utilisés ce qui représente $\binom{M}{3}$ triplets. Le principe de couplage partiel est de ne considérer qu'une sous ensemble des marginales au lieu de tous les histogrammes pour réduire les coûts de calculs. Plusieurs stratégies sont possibles mais doivent toutes contenir au moins une fois chaque variable. Dans notre étude, nous considérons 6 stratégies présentées dans la Table \ref{tab:triples}.
\begin{table}
    \footnotesize
    \legende{Stratégies de choix des triplets ($M = 10$)}
    \centering
    \begin{tabular}{|c|c|c|}
    \hline 
    Stratégie & \begin{tabular}[c]{@{}c@{}}Nombre de triplets\\ ($M = 10$)\end{tabular} & Triplets      \\ \hline 
    $+2$ & $5$         & {\scriptsize \triples{1}{2}{3}, \triples{3}{4}{5}, \triples{5}{6}{7},}                \\ 
     &          & {\triples{7}{8}{9} , \triples{9}{10}{1}}                \\ \hline
    $+1$ & $10$        & {\scriptsize \begin{tabular}[c]{@{}c@{}} \triples{1}{2}{3}, \triples{2}{3}{4}, $\ldots$ , \triples{8}{9}{10}\\ \triples{9}{10}{1}, \triples{10}{1}{2} \end{tabular}}          \\ \hline
    1/8 & $15 = 120/8$    & triplets choisis aléatoirement   \\ \hline
    1/4 & $30 = 120/4$    & triplets choisis aléatoirement   \\ \hline
    1/2 & $60 = 120/2$    & triplets choisis aléatoirement   \\ \hline
    1 & $120=\binom{10}{3}$ & tous les triplets (couplage total)\\ \hline
    \end{tabular}
    \label{tab:triples}
\end{table}

\subsection{Évaluation des performances}

Pour étudier les différentes stratégies de couplage, nous avons appliqué notre méthode aux stratégies de la Table \ref{tab:triples} avec des données synthétiques en dimension $M=10$. $R=20$ distributions gaussiennes furent générées aléatoirement et ajoutées ensemble avec des poids $\vect{\lambda}$ pour créer un histogramme $\tens{H}$ théorique. Nous avons ensuite généré plusieurs nombres d'échantillons $N = \left\{ 10^4, 3.10^4, 10^5, 3.10^5,10^6 \right\}$ pour enfin calculé les histogrammes 3D pour $K=30$. Concernant les paramètres de l'AO-ADMM Couplée, nous avons choisi $N_1 = 10^3$ itérations extérieures et $N_2 = 20$ intérieures avec un rang égal au nombre de composantes initial $R = 20$.

Pour analyser les performances de notre méthode, nous avons calculé l'erreur sur les marginales 1D et 3D pour chaque stratégie pour 10 réalisations différentes. \begin{equation}
    \label{eq:metricMarg1D}
    \Err{1D} = \sum\limits_{m=1}^M \left|\!\left| \sum\limits_{r=1}^R \vect{\lambda}_r\vect{a}^{(m)}_r - \sum\limits_{r=1}^R \reallywidehat{\vect{\lambda}_r}\reallywidehat{\vect{a}^{(m)}_r} \right|\!\right|^2
\end{equation} \begin{equation}
    \label{eq:metricMarg3D}
    \Err{3D} \!= \!\!\!\!\!\sum\limits_{i,j,k\in \mathcal{T}} \!\!\left|\!\left|\! \nset{\!\vect{\lambda},\!\factor{A}{i},\!\factor{A}{j},\!\factor{A}{k}\!} \!-\! \nset{\!\reallywidehat{\vect{\lambda}},\!\reallywidehat{\factor{A}{i}},\!\reallywidehat{\factor{A}{j}},\!\reallywidehat{\factor{A}{k}}\!} \!\right|\!\right|^2_F \!
\end{equation}
En Figure \ref{fig:errMarg}, les erreurs sur les marginales 1D et 3D sont représentées, montrant que le couplage '1/8' montre des performances similaires à la stratégie de couplage total. Cela montre que les stratégies de couplage partiel sont avantageuses en termes de coûts de calcul puisque la complexité de calcul est linéaire avec le nombre de triplets considérés. En outre, la différence d'estimation des marginales 3D est plus grande que celle des marginales 1D si on compare les différentes stratégies.

\begin{figure}
    \centering
    \begin{subfigure}{0.5\linewidth}
        \centering
        \includegraphics[width=\linewidth]{figures/errmarg1d.pdf}
        \label{fig:marg1d}
    \end{subfigure}%
    \begin{subfigure}{0.5\linewidth}
        \centering
        \includegraphics[width=\linewidth]{figures/errmarg3d.pdf}
        \label{fig:marg3d}
    \end{subfigure}
    \vspace{-0.5cm}
    \caption{\centering Évolution des erreurs pour les différentes stratégies}
    \label{fig:errMarg}
\end{figure}

\section{Application à la cytométrie en flux}

Afin de valider notre méthode, nous avons appliqué notre méthode à des données réelles controllées de CMF. 3 populations de cellules ont été teintes puis mélangées ($N = 10^5$ cellules) avant de passer dans un cytomètre où furent mesurés $M=4$ marqueurs. 3 marqueurs ont été associés aux 3 populations et un quatrième fut ajouté. Les 3 populations étaient des Lymphocytes B (LB) pour le marqueur CFSE, des Lymphocyte T pour le marqueur CD4 (LT) et des Macrophages (MP) pour CTV. Le marquage des LT n'ayant pas été optimal, nous n'avons pas pu obtenir une clair séparation entre les cellules CD4+ (LT) et les autres cellules (MP et LB). Les 3 populations furent mélangées selon 3 proportions différentes estimées manuellement par \textit{gating} (Figure \ref{fig:casdecole}). Les paramètres de l'expérience sont résumés dans la Table \ref{tab:pop}.
\begin{figure}
    \centerline{\includegraphics[width = 0.8\linewidth]{figures/cadecole.pdf}}
    \caption{\textit{Gating} des 3 populations. La figure montre 3 portes : P3 sépare les cellules CFSE+ (LB), P4 les cellules LT and P5 les cellules CTV+ (MP). Certaines cellules n'appartiennent à aucune porte.}
    \label{fig:casdecole}
\end{figure}
\begin{table}[]
    % \footnotesize
    \caption{\centering Propriétés des 3 populations de l'expérience controllée. + correspond à des fluorescences hautes et - des fluorescences basses.}
    \label{tab:pop}
    \centering
    \begin{tabular}{c||c|c|c|}
        \cline{2-4}
        & \multicolumn{3}{|c|}{Population de cellules} \\ \hline
        \multicolumn{1}{|c||}{\begin{tabular}[c]{@{}c@{}}Marqueur \\ Fluorescent\end{tabular}} & \begin{tabular}[c]{@{}c@{}}\!\!\!Macrophage\!\!\!\\ (MP)\end{tabular} & \begin{tabular}[c]{@{}c@{}}\!\!\!Lymphocyte B\!\!\!\\ (LB)\end{tabular} & \begin{tabular}[c]{@{}c@{}}\!\!\!Lymphocyte T\!\!\!\\ (LT)\end{tabular} \\ \hline
        \multicolumn{1}{|c||}{CD4}    & - & - & {\large \textbf{+-}} \\ \hline
        \multicolumn{1}{|c||}{CFSE}      & - & {\large \textbf{+}} & - \\ \hline
        \multicolumn{1}{|c||}{CTV}      & {\large \textbf{+}} & - & - \\ \hline
        \multicolumn{1}{|c||}{\!\!\!MHCII\!\!\!} & + & ++ & - \\ \hline 
    \end{tabular}
\end{table}

\subsection{Visualisation et clustering}

Après avoir appliqué notre méthode, nous obtenons $M$ matrices facteurs et $\vect{\lambda}$ ce qui représente toutes les propriétés de nos $R$ composantes. Pour interpréter au mieux un jeu de données, nous devons développer des outils de visualisations qui permettront aux utilisateurs d'extraire le plus d'informations.

De ce fait, le choix du nombre de composantes est crucial. Si $R$ est trop petit, les populations ne seront pas séparables. Il faut donc regrouper les composantes ayant des propriétés similaires, tout en prenant un grand nombre de composantes $R$. Nous procédons alors à un clustering hiérarchique (\textit{single linkage}) pour obtenir un dendrogramme. Les composantes sont alors groupées si leur distance \eqref{eq:distancemetacluster} est inférieure à une valeur de seuil.
\begin{equation}
    \label{eq:distancemetacluster}
    D(r,s) = 1-\prod\limits_{m=1}^M \left< \vect{a}^{(m)}_r,\vect{a}^{(m)}_s \right>
\end{equation}
Sur la Figure \ref{fig:casdecoleHD}, 3 groupes de composantes ont été obtenus et séparés par la procédure de clustering. Cela permet d'estimer la taille des 3 populations de cellules et de comparer notre méthode avec les résultats de \textit{gating}. Notre méthode est capable d'estimer une répartition où les populations rares comme les MP sont séparées grâce au clustering hiérarchique (Table \ref{tab:choiceR}). Les tailles des deux autres populations de cellules ont été estimées précisément. À noter que $R=60$ est au-dessus de la borne théorique d'identifiabilité de \cite{Kargas2018}, mais fourni tout de même une reconstruction précise, ce qui indique que les résultats d'identifiabilité peuvent être améliorer. Avec une stratégie de couplage partiel, nous avons été capables de retrouver des proportions du même ordre de grandeur mais moins précise que pour le couplage total (voir Table \ref{tab:coupling}).

\begin{table}[]
    \centering
    \caption{\centering Dépendence entre le choix du rang sur l'estimation de la proportion des macrophages (MP).}
    \begin{tabular}{|c|ccc|}
        \hline
        Méthode  & \multicolumn{3}{c|}{Proportion de Macrophage}               \\ \hline
        \textit{Gating}  & \multicolumn{1}{c|}{20.7\%} & \multicolumn{1}{c|}{8\%}  & 1.1\% \\ \hline
        $R = 20$ & \multicolumn{1}{c|}{20.2\%} & \multicolumn{1}{c|}{6.7\%} & 0.83\% \\
        $R = 40$ & \multicolumn{1}{c|}{20.1\%} & \multicolumn{1}{c|}{7.1\%} & 0.83\% \\
        $R = 60$ & \multicolumn{1}{c|}{20\%}  & \multicolumn{1}{c|}{7.7\%} & 0.91\% \\ \hline
    \end{tabular}
    \label{tab:choiceR}
\end{table}

\begin{table}[]
    \centering
    \caption{\centering Estimation de la proportion de Macrophage pour les stratégies de couplage partiel et total (4 vs 2 triplets).}
    \begin{tabular}{|c|ccc|}
        \hline
        Method      & \multicolumn{3}{c|}{Macrophage size}                         \\ \hline
        Gating      & \multicolumn{1}{c|}{20.7\%} & \multicolumn{1}{c|}{8\%}  & \multicolumn{1}{c|}{1.1\%} \\ \hline
        Fully coupled   & \multicolumn{1}{c|}{20\%}  & \multicolumn{1}{c|}{7.7\%} & 0.91\%           \\
        Partially coupled & \multicolumn{1}{c|}{17.2\%} & \multicolumn{1}{c|}{8.3\%} & 0.83\%           \\ \hline
    \end{tabular}
    \label{tab:coupling}
\end{table}

\begin{figure}[]
    \centering
    \includegraphics[width = \linewidth]{figures/casdecoleHD.pdf}
    \caption{CTFlowHD pour les données controllées contenant 3 populations. \textbf{En haut}: Matrices facteurs représentant les $M\!=\!4$ propriétés pour les $R\!=\!25$ composantes représentés par lignes. \textbf{Au milieu} : dendrogramme obtenu grâce au clustering hierarchique. \textbf{En bas} : taille de chaque composante en \%.}
    \label{fig:casdecoleHD}
\end{figure}

\section{Conclusion}

Nous proposons dans ce papier une nouvelle méthode probabiliste qui permet aux biologistes d'interpréter des données de cytométrie en flux sans perte de dimension lors des visualisations, a contrario avec les méthodes existantes. Notre méthode est capable de reconstruire des histogrammes en grandes dimensions et de les séparer en populations de cellules. Même si la factorisation couplée de tenseurs a une complexité de calcul polynomial, le couplage partiel permet d'aller encore plus loin en diminuant encore les temps de calculs tout en gardant des performances similaires d'estimation.

\bibliographystyle{IEEEbib}
\bibliography{refs}

\end{document}